\section{Introduction}

	\subsection{The Standard Model of Particle Physics}

    Central to the moden study of particle physics is the standard model,

    \begin{multline} \label{eqn:std-mod}
      L_{GWL} = \sum_{f} ( \bar{\Psi}_{f} ( i \gamma^\mu \partial \mu - m_{f} ) \Psi_{f} - eQ_{f} \bar{\Psi}_{f} \gamma^\mu \Psi_{f} A_{\mu} ) + \frac{g}{\sqrt{2}} \sum_{i} ( \bar{a}^i_L \gamma^\mu b^i_L W^+_\mu + \bar{b}^i_L \gamma^\mu a^i_L W^-_\mu )                        \\                           
              + \frac{g}{2x_w} \sum_f \bar{\Psi}_f \gamma^\mu ( I^3_f - 2s^2_w Q_f - I6e_f \gamma_5 ) \Psi_f Z_\mu - \frac{1}{4} | \partial_\mu A_v - \partial_v A_\mu - ie(W^-_\mu W^+_v - W^+_\mu W^-_v ) |^2                                         \\                                     
              - \frac{1}{2} | \partial_\mu W^+_v - \partial_v W^+_\mu - ie ( W^+_\mu A_v - W^+_v A_\mu ) + ig' c_w (W^+_\mu Z_v - W^+_v Z_\mu |^2 \\
              - \frac{1}{4} | \partial_\mu Z_v - \partial_v Z_\mu + ig' c_w (W^-_\mu W^+_v - W^+_\mu W^-_v ) |^2 - \frac{1}{2} M^2_\eta \eta^2  - \frac{gM^2_\eta}{8M_W} \eta^3  - \frac{g'^2 M^2_\eta}{32M_W}\eta^4    \\     
              + | M_W W^+_\mu + \frac{g}{2} \eta W^+_\mu |^2 + \frac{1}{2} | \partial_\mu \eta + i M_Z Z_\mu + \frac{ig}{2c_w} \eta Z_\mu |^2 - \sum_f \frac{g m_f}{2 M_W} \bar{\Psi}_f \Psi_f \eta.                                                                                
    \end{multline}

    The standard model, shown in equation \ref{eqn:std-mod}, is a quantum field theory that discribes the fundermental particles and how they interact.
    While this report does require, or attempt, a detailed understanding the intricate detail of the stardard model;
    the aim of many particle physics experiments is to varify, measure and expand the model.
    Dispite being the current best theory to explain particle interactions, the model is not complete.
    There are many undescribed phemomina, such as the matter domination in the universe, that require physics behond the standard model in order to be described.
    To that end, major international efforts, namely in the form of the Large Hardron Collider, aim to gain further knowledge and understanding of the underlying physics of the universe. \cite{ref:std}

  \subsection{Field Programable Gate Arrays}

  \subsection{The LHCb Experiment}

    One experiment at the Large Hadrom Colider is Large Hadron Colider beauty (LHCb).
    Located at intersection point 8, LHCb is designed to study rare particly physics phemonena, such as lepton flavour violation and CP violation. 
    The decays studied in the LCHb are via exotic hadronic decays of Bottom or Charm quarks that form sort lived hardons. 
    These hardons, commonly B mesons, travel in the order of a few cm in the detector before decaysing. As such, B meson decays can be identied by decay products that propogate via a secondary vertex.

    \begin{figure}[h!]
      \centering
      \includegraphics[scale=0.5]{LHCb_Det.jpg}
      \caption{The LHCb Detector along the bending plane.}
      \label{fig:LHCb_Collab}
    \end{figure}\FloatBarrier

    As B mesons are light (in comparision to other particles studied in the LHC), the decays products are produced at a shallow angle relivite the the beam pipe;
    this is the driving factor in the design of the exeperiment. 
    LHCb is a single arm forward spectormeter.
    Surrounding the point of collision is the VErtex LOcator (VELO), this high precision detector uses silicon strips to detect ionising particles as they propogate from a collition and provides the coordinates of the particle in terms of R\footnote{Radial distance from the beam pipe.} and $\phi$\footnote{Asumthal angle.}.
    By reconstructing the paths of partics back to the intersection point, it can be identified wether or not the particular decay practicles are a product of the primary vertex\footnote{The position at which the protons collided.}, or a secondary vertex\footnote{The decay point of a short lived particle. i.e. B Meson.}.
    \par
    The Rich dectector, comprised of two subdectectors eitherside of the magnet, uses cherincov radiation to deduce the velocity of the particle. The silicon trackers, labeled TT and T1-3 in Figure~\ref{fig:LHCb_Collab}, calculate the angle deflection by the magnet. Be combining the velocity and angle of deflection, the mass, momentum and energy of the particles can be decuced from simple relitivistic kinematics.
    \par
    The meuon detectors, labeled M1-5 in Figure~\ref{fig:LHCb_Collab}, are important to detect muon's the detector. 
    This is of particular importance on LHCb as muons can be easily missidentified as charged pions, due to there simular mass.
    \par
    HCAl and ECAL, shown in Figure~\ref{fig:LHCb_Collab}, are hadronic and electric calorimeters respectively. 
    Both measure the total energy of incomming particles.
    As the calorimeters are absorbing of the particles they detect, any leptonic particle reaching the M2-5 muon detectors can be assumed to be a muon.
    Electrons and Photons are absorbed by the ECAl and any Tauons would have decayed long before reaching the far muon detectors.

      \subsubsection{VELO Upgrade}



      \subsubsection{The Role of FPGA's in the VELO Upgrade}

      
